\documentclass[a4paper,10pt]{report}
\usepackage[utf8x]{inputenc}
\usepackage[french]{babel}
\usepackage[letterpaper]{geometry}
\geometry{verbose,tmargin=2cm,bmargin=2cm,lmargin=2cm,rmargin=2cm}
\usepackage{graphicx} 
\usepackage{color}
\setcounter{secnumdepth}{3}
\setcounter{tocdepth}{3}
\usepackage{amsmath}
\usepackage{amssymb}
\usepackage{mathptmx}
\usepackage{ifthen}
\usepackage[T1]{fontenc}
\usepackage{float}
\usepackage{textcomp}
\usepackage{setspace}
\usepackage{listings}
\usepackage{array}
\usepackage[unicode=true, pdfusetitle, bookmarks=true,bookmarksnumbered=false,bookmarksopen=false,
breaklinks=false,pdfborder={0 0 0},backref=false,colorlinks=false,pdfauthor={St\'ephanie Ouillon -- Tiezhen Wang}]{hyperref}

\definecolor{colKeys}{rgb}{0,0,1} 
\definecolor{colIdentifier}{rgb}{0,0,0} 
\definecolor{colComments}{rgb}{0,0.5,1} 
\definecolor{colString}{rgb}{0.6,0.1,0.1} 

\lstset{%configuration de listings 
float=hbp,% 
basicstyle=\ttfamily\small, % 
identifierstyle=\color{colIdentifier}, % 
keywordstyle=\color{colKeys}, % 
stringstyle=\color{colString}, % 
commentstyle=\color{colComments}, % 
columns=flexible, % 
tabsize=2, % 
frame=trBL, % 
frameround=tttt, % 
extendedchars=true, % 
showspaces=false, % 
showstringspaces=false, % 
numbers=left, % 
numberstyle=\tiny, % 
breaklines=true, % 
breakautoindent=true, % 
captionpos=b,% 
xrightmargin=+0cm, % 
xleftmargin=+0cm
}

\AtBeginDocument{
  \def\labelitemi{\(\bullet\)}
}

% Title Page
\title{Integration of the "trains protocol" to the middleware JGroups \\ Projet de fin d'\'etude}
\author{St\'ephanie Ouillon -- Tiezhen Wang \\ \includegraphics[scale=0.3]{img/int.jpg}}
\date{\today}


\begin{document}
\maketitle

\tableofcontents

\chapter{Introduction}

The trains protocol is a uniform and totally-ordered broadcast protocol \footnote{ Défago, X., Schiper, A. et Urbán, P. (2004). Total order broadcast and multicast algorithms : Taxonomy and survey. ACM Comput. Surv., 36:372?421}.
It is designed to be a throughput-efficient protocol, especially for short messages (100 bytes or lower)\footnote{M. Simatic. Communication et partage de données dans les systèmes répartis (Data communication and data sharing in distributed system, in French). PhD thesis, École Doctorale ÉDITE, October 2012 (available at http://www-public.it-sudparis.eu/~simatic/Recherche/Publications/theseSimatic.pdf)}.
So far, the trains protocol has been implemented in C during the year 2012 at T\'el\'ecom SudParis.

\section{Integration with JGroups}

JGroups\cite{jgroups} is a reliable multicast system written in Java language. It has a flexible protocol stack which allows developpers to 
adapt it to their application requirements.
The primary goal of this project is to provide JGroups with an implementation of the trains protocol.

JGroups adds a "grouping" layer over a transport protocol, internally keeping a list of participants. 
It can be used to create groups of processes whose members can send messages to each other\cite{wikipedia}.
More importantly, when a member joins or leaves, JGroups takes care of transfering the current state of the group so that the new member
gets the history of the messages that were sent while it was offline.

Because the initial C implementation of the trains protocol doesn't implement this state transfer, integrating the protocol to JGroups
would allow to test the default state transfer protocol of JGroups with the protocol. 

\section{Global Roadmap}

The steps of this projet will be the following ones: 

\begin{enumerate}
  \item \textbf{Java implementation:} the trains protocol needs to be implemented in Java in order to be integrated to JGroups
  \item \textbf{Integration in the protocol stack of JGroups}
  \item \textbf{State transfer}: testing the state transfer feature of JGroups with the trains protocol
  \item \textbf{Performance and application:} using JGroups and the trains protocol in a real use-case and study performances
\end{enumerate}

\section{A closer look at the trains protocol}

\subsection{Brief introduction to the protocol}

\textit{The following part is freely inspired by the chapter 4 entitled "The trains protocol" of the thesis of Michel Simatic (in French)\footnote{\textit{Ibid}}}.\\

The trains protocol comes from the idea of the train protocol of [Cristien, 1991]. Briefly explained, one or several trains of messages
run between the participating processes which are distributed on a virtual ring. The following sections approximatively exposes some of the
notions explained in the thesis.\\


\textbf{System model:} The trains protocol is meant to be run on a small cluster of homogeneous machines connected on a local network (typically with a switch). Each machine
can run one or several processes participating to the trains protocol. These processes are distributed on a virtual ring. Each processus is then
connected via TCP to the previous and the following process on this ring. The messages run in the same way.
Given the simplicity of the implemented communications, the homogeneous environment and the low latency of the network, the processus at the other end of the connection is considered to be failing if the TCP connection fails.\\


\textbf{Some definitions:}\\ 
A \textit{circuit of trains} is the virtual ring on which the processes participating to the 
trains protocol are distributed.\\
The \textit{train} is a token running on the circuit. It carries the messages to be broadcasted inside \textit{wagons}.
When a processus wants to broadcast a message, it adds it to a wagon (a data structure) and then waits to receive the first passing train/token to add
this wagon at the end of the train.\\
The \textit{address} of a process is a unique identifiant for the given process in the trains protocol.

\subsection{State of the implementation in C}

The trains protocol is completely implemented in C - except the feature of state transfer - and some performance tests have been made.
Documentation is available through doxygen.
What is important to remember is that this version of the trains protocol has been developped and tested on Linux.

The user can use the trains protocol by loading the libtrains.so library and interact with a set of functions:
\begin{itemize}
  \item interface.c: trInit() to initialize the process, trTerminate()...
  \item applicationMessage.c: newmsg() to create a new message, utoBroadcast() to send it
\end{itemize}



\chapter{Java implementation of the trains protocol}
\section{Using JNI}
\section{Designing the API}
\section{Modifications of the initial C API}
\section{Towards a cross-plateform code}
\section{Feedback: what is good to know with JNI}

\chapter{JGroups}
\section{}


\chapter{Conclusion}

\section{}

\section{What is yet to be done}

\chapter{Manual}


\listoffigures

\begin{thebibliography}{99}
  \bibitem{jgroup} \href{http://www.jgroups.org/}{Le site de JGroups~: http://www.jgroups.org/}
\end{thebibliography}

\end{document}          
