\documentclass[a4paper,10pt]{report}
\usepackage[utf8x]{inputenc}
\usepackage[french]{babel}
\usepackage[letterpaper]{geometry}
\geometry{verbose,tmargin=2cm,bmargin=2cm,lmargin=2cm,rmargin=2cm}
\usepackage{graphicx} 
\usepackage{color}
\setcounter{secnumdepth}{3}
\setcounter{tocdepth}{3}
\usepackage{amsmath}
\usepackage{amssymb}
\usepackage{mathptmx}
\usepackage{ifthen}
\usepackage[T1]{fontenc}
\usepackage{float}
\usepackage{textcomp}
\usepackage{setspace}
\usepackage{listings}
\usepackage{array}
\usepackage[unicode=true, pdfusetitle, bookmarks=true,bookmarksnumbered=false,bookmarksopen=false,
breaklinks=false,pdfborder={0 0 0},backref=false,colorlinks=false,pdfauthor={St\'ephanie Ouillon -- Tiezhen Wang}]{hyperref}

\definecolor{colKeys}{rgb}{0,0,1} 
\definecolor{colIdentifier}{rgb}{0,0,0} 
\definecolor{colComments}{rgb}{0,0.5,1} 
\definecolor{colString}{rgb}{0.6,0.1,0.1} 

\lstset{%configuration de listings 
float=hbp,% 
basicstyle=\ttfamily\small, % 
identifierstyle=\color{colIdentifier}, % 
keywordstyle=\color{colKeys}, % 
stringstyle=\color{colString}, % 
commentstyle=\color{colComments}, % 
columns=flexible, % 
tabsize=2, % 
frame=trBL, % 
frameround=tttt, % 
extendedchars=true, % 
showspaces=false, % 
showstringspaces=false, % 
numbers=left, % 
numberstyle=\tiny, % 
breaklines=true, % 
breakautoindent=true, % 
captionpos=b,% 
xrightmargin=+0cm, % 
xleftmargin=+0cm
}

\AtBeginDocument{
  \def\labelitemi{\(\bullet\)}
}

% Title Page
\title{Integration of the "trains protocol" to the middleware JGroups \\ Projet de fin d\'etude}
\author{St\'ephanie Ouillon -- Tiezhen Wang \\ \includegraphics[scale=0.3]{img/int.jpg}}
\date{\today}


\begin{document}
\maketitle

\tableofcontents

\chapter*{Introduction}

\chapter{Goals of the project}

\section{The trains protocol}
\subsection{Brief introduction to the protocol}
\subsection{State of the implementation in C}

\section{Java implementation of the trains protocol}
\subsection{Using JNI}
\subsection{Designing the API}
\subsection{Modifications of the initial C API}
\subsection{Towards a cross-plateform code}
\subsection{Feedback: what is good to know with JNI}

\section{JGroups}
\subsection{}


\subsection{Architecture standard}

Un jeu standard fonction de la manière suivante~:
\begin{enumerate}
  \item Le joueur 1 se connecte et crée son personnage (un objet local)
  \item Le joueur 2 se connecte et crée lui aussi son personnage
  \item Le joueur 1, désirant attaquer le joueur 2, envoie à celui-ci une information lui signalant qu'il vient de l'attaquer avec x points de dégât
  \item Le joueur 2 recevant cette information, met à jour son personnage (objet local) et renvoie au joueur 1 les informations dont il a besoin pour continuer de jouer
\end{enumerate}

\begin{figure}[!hb]
  \begin{center}
    \includegraphics[scale=0.3]{gasp_ex1.png}
  \end{center}
  \caption{\label{gasp_ex1}Architecture standard}
\end{figure}

\subsection{Architecture répartie}

\section{Middleware de communication}

Pour cela, nous allons analyser et comparer les points forts, les points faibles, et les points qui nous paraissent important des 3 APIs suivantes~:
\begin{itemize}
  \item JGroups~\cite{jgroup}
  \item Spread~\cite{spread}
  \item Appia~\cite{appia}
\end{itemize}


\section{L'API}

\part{Comparaison des différents middleware \\ Choix de l'un d'entre eux}

\chapter{Spread}

\section{Résumé des fonctionnalités}

\section{Facilité d'installation et de mise en œuvre des exemples}

Ce point d'analyse du middleware est excellent. Il y a~:
\begin{description}
  \item[Un \emph{Userguide} :] Il est disponible pour chacune des API en fonction du langage que l'on utilise.
  \item[Une F.A.Q.~:] Une F.A.Q.\footnote{Foire Aux Questions} qui permet d'avoir les réponses aux questions de base.
  \item[Une mailing~list :] Celle-ci est à jour, et bien utilisée.
  \item[Un bug tracker :] On peut signaler un bug par mail à partir du site aux développeurs.
  \item[Un dépot svn :] Il permet de télécharger des sources à jour, en plus d'avoir accès aux différentes versions.
\end{description}



L'API spread est disponible dans plusieurs langages différents (dont java et C$/$C$+$$+$). Elle est donc utilisable à la fois sous téléphone Android, mais aussi sous iPhone.\\
Il existe un site web pour le portage de l'API spread sur Android \href{http://spreadAndroid.com}{(http://spreadAndroid.com)}, mais celui-ci est vide à l'heure actuelle.\\
De plus, étant donné qu'elle est implémenté par-dessus la couche TCP$/$IP\footnote{Transmission Control Protocol $/$ Internet Protocole}, elle peut fonctionner sur un réseau \emph{ad hoc}.


\section{Disponibilité sur Android, iPhone et réseau \emph{ad hoc}}

\begin{table}[!hbp]
  \begin{centering}
    \begin{tabular}{|>{\centering}m{4cm}|>{\centering}m{4cm}|>{\centering}m{4cm}|>{\centering}m{4cm}|}
      \hline 
      Type de Test \textbackslash{} API & Spread & Appia & JGroups\tabularnewline
      \hline
      \hline 
      Fonctionnalités & \multicolumn{3}{c|}{API de communication}\tabularnewline
      \hline 
      Facilité d'installation & Serveur facile \\ Client plus complexe & Simple à partir des jars & Simple à partir des jars \\ Bien configurer le classpath avec le code source\tabularnewline
      \hline
    \end{tabular}
  \par\end{centering}
\caption{\label{tab_recap}tableau récapitulatif}
\end{table}


Le support du multicast sous Android reste pous nous un mystère, après une longue investigation il en ressort que depuis Android 1.5 le multicast est censé être supporté officiellement.
\lstset{language=java} 

Il existe une classe dans l'API Android \lstinline{java.net.MulticastSocket} présente dès la première version de l'API.

\lstset{commentstyle=\textit} 
\begin{lstlisting}
       MulticastLock lock;
       WifiManager wifi = (WifiManager)getSystemService(Context.WIFI_SERVICE);
       lock = wifi.createMulticastLock("mylock");
       lock.setReferenceCounted(true);
       lock.acquire();
\end{lstlisting} 

Une vérification dans le /proc/config.gz du téléphone (fichier .config lors de la compilation du noyau Linux) nous indique : \\
\# cat config $|$ grep MULTIC\\
\# CONFIG\_IP\_MULTICAST is not set

Vous pouvez télécharger l'application en utilisant le QRCode de la figure \ref{touchsurface} de la page \pageref{touchsurface}, ou en allant sur le site de Yann~\cite{site_yann}.

\part{Bilan}

\chapter{Conclusion}

\section{Les objectifs atteints}

\section{Le travail restant}

\chapter{Manuel}


\listoffigures

\begin{thebibliography}{99}
  \bibitem{jgroup} \href{http://www.jgroups.org/}{Le site de JGroups~: http://www.jgroups.org/}
  \bibitem{spread} \href{http://spread.org}{Le site de Spread~: http://spread.org}
  \bibitem{key1} \href{https://github.com/fallen/JGroupsRemoteObjectAPI}{Lien vers le code source de la Remote Object API~: https://github.com/fallen/JGroupsRemoteObjectAPI}
\end{thebibliography}

\end{document}          
